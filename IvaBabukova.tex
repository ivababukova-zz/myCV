\documentclass{tccv}
\usepackage[english]{babel}
\usepackage{mathtools}
\usepackage{relsize}
\begin{document}
\setlength{\emergencystretch}{3em}
\part{Iva Babukova}

%
\personal
    {36 Garriochmill road \newline G20 6LT Glasgow}
    {+44 075842 82034}
    {ibabukova@gmail.com}

\section{Education}

\begin{yearlist}

\item[University of Glasgow]
     {2012 -- 2016}
     {Computer Science \& Mathematics}
     {Glasgow, UK}

\item[Sofia Mathematics High School]
     {2007 -- 2012}
     {Advanced study: Mathematics, German language, Computing}
     {Sofia, Bulgaria}

\item[107 Secondary school "Khan Krum"]
     {2003 -- 2007}
     {Advanced study: Mathematics}
     {Sofia, Bulgaria}

\item[121 Primary scholl]
     {2000 -- 2003}
     {Sofia, Bulgaria}

\end{yearlist}

\section{Skillset}

\begin{skillist}
\item{Organiser} {In May 2013 I have co-founded GUTS -- \href{http://gutechsoc.com}{Glasgow University Tech Society}. Currently I am a treasurer in the society, which promotes collaboration, innovation and excellence in technology among GU students by organising talks from academia and industry and workshops on new technologies. My biggest achievement within the society was co-organisation of a \href{http://storify.com/Eventhread/gu-hackaton}{hackathon}, which has attracted about 50 participants and over a dozen of corporate sponsors.}
\item{Team player} {I have been part of a team many times. For example, last month when I attended the "Code for good" hackathon, organised by JP Morgan and the Barclays Openminds hackathon, both in London. For really short preiod of time: around 24 hours for each hackthon we were supposed to produce a working solution on a given problem. During the sleepless hours of cooperating with the other four members of the team, working under preasure I learned how important efficient teamwork is. From my point of view the core success of Agile Software Development is people, communicating with each other, working together.}
\item{Iniciative and creative problem solving}{As an elected student representative of Mathematics and Algorithmic Foundation classes, my responsibility is to be the bridge between the students and the teaching office.For example, there was recently a problem with one of the professors: he does not speak loud enough, nobody could hear him. I couldn't go to him and tell him that he speaks too quiet, it would be really impolite and it would not solve the problem. Few days later I had lecture in the same room with other professor. What I found out was that the acoustic of the room was very poor, because I could not hear him properly as well. I went to the teaching office and aksed them to change the venue where the lectures are held. After the venue was changed, the problem was solved.} 
\item{Self-motivation and drive to suceed}{I have seen examples in my life when really smart and talented people do not accomplish well in life, because they cannot motivate themselves. Since my school years mainly teachers have noticed that I am really ambitious and motivated to pursue my aims. On my early age I has really interested in mathematics. I used to attend extra courses and competitions for talented students and I used to spend hours everyday in solving maths questions. That made my mathematics skills really strong which helps me a lot in programming. One of my hobbies are martial arts. I used to train Taekwondo almost every day for three years before I started university. I was one of the competitors of the fighting club. Every win was a result of hard work and patience. If I wasn't motivated, I would never be able to succeed. }
\item{Effective communication}{Communication is about more than just exchanging information; it's also about understanding the emotion behind the information. I used to be a mathematics tutor in a secondary school for seven months. There I found out that I possess strong communication skills like emotional awareness, listening, non-verbal communication and managing stress. This job helped me to upgrade them even more. First of all, I learned to explain in clearer and more coherent manner so children could understand even the hardest part of the lesson. Second, I found out that body language is really important. I could make students be interested and more concentrated in what I was speaking using the right tone of my voice, body movement and gestures and eye contact. Also, my listening skill and patience became superior.} 
\item{Adaptability}{My life demands high adaptability skills. For more than one year I have been living in Glasgow, 4000km away from home. Last year I lived in a student accommodation with people from various places around the world. I was exposed to completely new environment (the climate, main language, people,etc were dissimilar from what have been used to). I was so happy that I didn't have any difficulties with handling the changes. I managed to adapt and very quickly I started feeling like being at home. I am used to travelling a lot, because recently I was accepted to take part in some programming events around UK. I really enjoy that, because I have the chance to expand my horizon by visiting new interesting places and meeting new people with same interests as me.}
\end{skillist}

\section{Projects}

%\newcommand{printfriendly}

\begin{eventlist}
\item{Stock Exchange}
     {Scalable stock exchange}
     {This is a team effort undertaken during Barclay's hackathon in Autumn 2013. Along with 3 other participants I have managed to create a website that represents a scalable stock exchange, developed by one of my team-mates. The code shows a genuine 24-hour effort, and no post-hackathon development was carried out. I have plans to carry on with the project, depending on other commitments.}
     {github.com/ivababukova/StockExchange}
\item{The Food Chain}
     {Mobile application development}
     {This is again a team effort, this time undertaken during JP Morgan's hackathon in Autumn 2013. I was working with 4 other participants and I have managed to carry out the frontend, cooperating with one of my team-mates, while the others were supposed to work with the backend. For 24 hours I and my team-mate managed to develop an android application which enchances more efficient food delivery to the customers.}
    {IvaBabukova@bitbucket.org/jmcghee/codeforgood-app.git}
\end{eventlist}

\section{Toolset}

\begin{skillist}

\item{Python}
     {I used to study Python during my first year at university. This language really impressed me with its design phylosophy: it emphasizes code readability and its syntax allows programmers to express concepts in fewer lines of code, unlikely other languages (Java for example). I really enjoy writing code in Python, especially developing games using Pygames.}

\item{C++}{At school I used to learn C++ three years. I used to implement basic algorithms, solving problems during programming classes, writing code in C++ sometimes at home just for fun. I find it really powerful language, although I don't use it very often nowadays. }

\item{Java}{I like to use Java for large projects, due to several reasons. It is intuitively to debug Java program, especially if one uses Eclipse. Java is Object-Oriented language and this allows one to create modular programs and reusable code.}

\item{Android App Development, Web Development}{ I have self-taught in the basics of android applications development and I have developed websites using JavaScript, CSS, HTML. This is really interesting and useful part of computing nowadays which I would like to continue studying in more depth.}
\end{skillist}


\end{document}
